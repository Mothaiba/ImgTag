\documentclass[10pt,twocolumn,letterpaper]{article}
\usepackage[english]{babel}
\usepackage{cvpr}
\usepackage{times}
\usepackage{epsfig}
\usepackage{graphicx}
\usepackage{amsmath}
\usepackage{amssymb}

\usepackage[pagebackref=true,breaklinks=true,letterpaper=true,colorlinks,bookmarks=false]{hyperref}

\def\cvprPaperID{12345}
\def\httilde{\mbox{\tt\raisebox{-.5ex}{\symbol{126}}}}

\ifcvprfinal\pagestyle{empty}\fi

\begin{document}

\title{Write an article with Tex}

\author{au1\\
Insti1\\
Addre1\\
{\tt\small au1@gmail.com}
\and
au2\\
Insti2\\
Addre2\\
{\tt\small au2@gmail.com}
}

\maketitle


\begin{abstract}
This article proposes a novel method to write a paper.
\end{abstract}

\section{Introduction}
Since I don't know how to write an article, this method is novel to me.\cite{DBLP:journals/corr/RajpurkarMKCTWN15}

\section{Main section}
Write these codes

\section{Summary and Future work}
\subsection{Summary}
This method is efficient
\subsection{Future work}
Try to add more features


\nocite{*}
\bibliographystyle{plain}
\bibliography{Andrew}


\end{document}